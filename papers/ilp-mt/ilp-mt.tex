\documentclass[english,11pt]{article}
\usepackage{naaclhlt2009}
\usepackage{times}
\usepackage{latexsym}
\usepackage{amsthm} 
\usepackage{amsmath}
\usepackage{amssymb}
\usepackage[authoryear]{natbib}
\usepackage{multirow}
\usepackage{subfig}
\usepackage{url}

\setlength\titlebox{6.5cm}    % Expanding the titlebox

\theoremstyle{plain}
\theoremstyle{plain} 
\newtheorem{thm}{Theorem}
  \theoremstyle{plain}
  \newtheorem{algorithm}[thm]{Algorithm}

\usepackage{babel}

%% sloppy linebreaks
%\sloppy

%% no extra spacing after dots
%\frenchspacing

%% interline spacing
\renewcommand{\baselinestretch}{.96}

% just a working title
\title{Revisiting Optimal Decoding for IBM Machine Translation Model 4}

% \author{}

\date{}

\begin{document}
\maketitle
\begin{abstract}
  This paper revisits optimal decoding for statistical machine
  translation using IBM Model 4.  We show that exact/optimal inference
  using Integer Linear Programming is more practical than previously
  suggested when used in conjunction with the Cutting-Plane Algorithm.
  In our experiments we see that exact inference can provide a gain of
  up to one BLEU point for sentences of length up to 30 tokens.
\end{abstract}

\section{Introduction}
\label{sec:introduction}
The search for the most likely translation (aka decoding, search,
MAP inference) given a statistical MT model has traditionally been
peformed by greedy or beam-based methods/heuristics {[}site some{]}.
While being efficient, most of these methods (do we know enough about
decoding to make any claims here?) have two drawbacks: (1) they are
approximate and give no bounds as to how far their solution is away
from the true optimum---this makes it sometimes hard to tell whether
the model or the search algorithm need to be improved; (2) they don't
scale (development-wise, not efficience) well with additional global
constraints. 

By constrast, recently researchers in other NLP appliciation areas
have increasingly relied on Integer Linear Programming as means to
find MAP solutions. ILP overcomes the two drawbacks mentioned above
as is guaranteed to be exact, and can be easily used to enforce global
constraints through additional linear inequalities. However, guaranteed
exactness, and generic global inference comes usually at the price
of efficiency. 

In fact, ILP has also been used as a method for MT decoding with IBM
Model 4 {[}cite Germann{]}. In {[}cite{]} work it became clear that
a naive ILP-based does not scale up to more than simple short sentences
due to an exponential number of constraints necessary to represent
the decoding problem in ILP. However, recent work in Dependency Parsing
{[}cite you and me{]} showed that ILP can still be efficient for very
large prograns when used in an incremental fashion. This raises the
question whether incremental/Cutting Plane ILP can also be used for
decoding of Model 4 for real world sentences. 

In this work we show that it is indeed possible to decode Model 4
with incremental ILP, at least for sentences to up to 30(? we have
to be more defensive here I think) words, and a simple two-gramm languange
model. This allowed us to give a definite answer as to how good/bad
Model 4 actually is, and how good/bad its state-of-the-art rewrite
decoder {[}cite{]}. Exact inference increases Bleu score by about
1 point for two language pairs when compared to the results of the
rewrite decoder {[}this is not really the answer{]}. From this we
conclude that the rewrite decoder indeed performs well, but still
can be slightly improved. (At the price of efficiency, that is)

Moreover, we believe that our work can be the basis of exciting and
really cool and also sexy future work. For example, as mentioned ILP
allows for a principled and declarative implementation of global constraints
and hence we may ask whether Model 4 can be improved through additional
global linguistic constraints; can we map other MT problems to an
ILP representation, and will this get slower, or maybe faster {[}why
could it{]}? How can larger language models be handled? (such models
are relatively easy to incorporate in left to right or greedy approaches
but pose a serious problem in the ILP sense). 

Finally, we observed that preprocessing and generating of ILPs is
as cpu-intensive as actual inference. This leads to the more general
question of how to integrate model construction more tightly with
the ILP solver (maybe by using Pricing strategies).
% we could wrap a supersection around the following 3 sections, but that 
% takes extra space. 

\section{IBM Model 4} 
\label{sec:ibm-model-4}
In this paper we focus on the translation model defined by IBM Model 4
(TODO:Cite).  IBM Model 4 models the translation process as a
generative story of how a sequence of target words (in our case French
or German) is generated from a sequence of source words (English).

The generative story is as follows.  Imagine we have an English
sentence, $\mathbf{e} = e_1, \dots,e_l$ and $e_o$ is the NULL word and
French sentence, $\mathbf{f} = f_1, \dots, f_m$.  First a fertility is
drawn for each English word (including the NULL symbol).  Then, for
each $e_i$ we then independently a number of French words equal to
$e_i$'s fertility.  Finally we process the English source tokens in
sequence to determine the positions of their generated French target
words.  We refer the read to (TODO:Cite) for full details.

Translation using IBM Model 4 is performed by treating the translation
process a noisy-channel model where the probability of the source
sentence given a target sentence is, $P(\mathbf{s}|\mathbf{t}) =
P(\mathbf{t}|\mathbf{s}) \cdot P(\mathbf{s})$, where $P(\mathbf{s}$) is
a language model of the source language (English).

%%% Local Variables: 
%%% mode: latex
%%% TeX-master: "ilp-mt"
%%% End: 
 
 
\section{Integer Linear Programming Formulation}
\label{sec:ilp}

\global\long\def\source{\mathbf{e}}
\global\long\def\target{\mathbf{f}}
\global\long\def\align{\mathbf{a}}
\global\long\def\start{\text{START}}
\global\long\def\stop{\text{END}}
\global\long\def\null{\text{NULL}}
\global\long\def\sourceset{S}


Given a trained IBM model 4, and a French sentence $\target$ we need
to find the English sentence $\source$ and alignment $\align$ with
maximal $p\left(\align,\source|f\right)\backsimeq p\left(\source\right)\cdot p\left(\align,\target|\source\right)$.
\citet{germann01fast} presented an Integer Linear Programming~(ILP) formulation of this problem. In this section we will give a very high-level description of this ILP formulation.%
\footnote{Note that our formulation differs slightly because we use a first order modelling
language that imposed certain restrictions on the type of constraints
allowed.%
} For brevity we refer the reader to the original work for details of the ILP formulation. 

In their ILP formulation an English translation is represented as a path through a set of English candidate tokens. A set of binary variables denote whether or not certain token pairs are directly connected through this path. Among constraints which guarantee that each French word has exactly one English word that generates it, the program also contains an exponential number of constraints that forbid each possible cycle the variables can represent. It is this set of constraints that renders decoding with ILP difficult. 

 





\section{Cutting Plane Algorithm}
\label{sec:cutting-plane}
\global\long\def\y{\mathbf{y}}


The Integer Linear Program we have described above has an exponential
number of (cycle) constraints. Hence, simply passing the ILP to an
off-the-shelf ILP solver is not practical for all but the smallest
sentences. For this reason the original {}``Optimal Decoding'' work
only considers sentences up to a length of 8 words. However, recent
work~\citep{riedel06incremental} has shown that even exponentially
large decoding/search problems can efficiently solved using ILP solvers
if a so-called Cutting-Plane Algorithm~\citep{dantzig54solution}
is used.%
\footnote{It is worth mentioning that Cutting Plane Algorithms have been successfully
applied for solving very large instances of the Travelling Salesman
Problem, a problem essentially equivalent to the decoding in Model
4 {[}cite someone?.%
} 
% In the following we will present a Cutting Plane algorithm for IBM
% Model 4:
% \begin{enumerate}
% \item Construct ILP $I$ without cycle constraints

% \begin{enumerate}
% \item \textbf{do}

% \begin{enumerate}
% \item solve $I$ and assign to solution $\y$
% \item find cyclic paths in solution $\y$
% \item add corresponding cycle constraints to $I$
% \end{enumerate}
% \textbf{until} no more cyclic paths can be found

% \item \textbf{return} $\y$
% \end{enumerate}
% \end{enumerate}
In our case the Cutting Plane algorithm starts with a subset of the complete set
of constraints, namely all constraints but the (exponentially many)
cycle constraints. The corresponding ILP is solved by a standard ILP
solver, and the solution is inspected for cycles. If it contains
no cycles we are done (we have found the true optimum: the solution
with highest score that does not violate any constraints). If the
solution does contain cycles, the corresponding constraints are added
to the ILP which is in turn solved again. This process is continued
until no more cycles can be found. 

% It is difficult to make claims about a guaranteed worst-case runtime
% (or number of iterations) of this algorithm. However, if the linear
% scoring function (in other words, the translation model and language
% model parameters) already provides a preference for cycle-free solutions,
% we can expect this algorithm to be efficient. For example, if we assume
% that the translation/distortion model has a very strong preference
% for monotonic solutions then clearly the highest scoring solution
% is likely to be cycle-free.




%%% Local Variables: 
%%% mode: latex
%%% TeX-master: "ilp-mt"
%%% End: 


\section{Evaluation}
\label{sec:evaluation}
In this section we describe our experimental setup and results.

\subsection{Experimental setup}
\label{sec:experimental-setup}

Our experimental setup is designed to answer several questions: (1) Is
exact inference in IBM Model 4 possible for sentences of moderate
length? (2) How fast is exact inference using the cutting plane
method? (3) How well does the ReWrite decoder perform in terms of
finding the optimal solution? (4) Does exact decoding increase provide
better translations?

In order to answer these questions we obtain a trained IBM Model 4 for
French-English and German-English on Europarl version 3 using GIZA++.
A bigram language model with Witten-Bell smoothing was built using the
CMU-Cambridge Language Modeling Toolkit.

For exact decoding we use the two models to generate ILP programs for
sentences of up to (and including) length 30 tokens for French and 25
tokens for German\footnote{These limits were imposed to ensure the
  Python script generating the ILP programs did not run out of
  memory.}.  We filter translation candidates in a similar manner to
(TODO:CITE) by using only the top ten translations for each
word\footnote{Based on $t(e|f)$.} and a list of zero fertility
words\footnote{Extracted using the rules in the filter script
  \texttt{rewrite.mkZeroFert.perl}}.  This resulted in 1101 French
sentences and 1062 German sentences for testing purposes.  The ILP
programs were then solved using the method described in
Section~\ref{sec:ilp}.  The same models were used with the ISI ReWrite
Decoder to solve the same set of sentences.


\subsection{Results}
\label{sec:results-results}

\begin{table*}[t]
  \centering
  \subfloat[French-English\label{tab:results:french}] {
    \footnotesize
  \centering
  \begin{tabular}{|l|r|r|r|r|r|r|r|r|}
    \hline
    \multirow{2}{*}{Length} & \multirow{2}{*}{\#} & \multicolumn{4}{|c|}{Solve Stats} & \multicolumn{3}{|c|}{BLEU} \\
    & & \% eq & \% gt & \%err & ST & ReW & ILP & Diff \\
    \hline
    1--5   & 21  & 52.4 & 47.6 & 6.9 & 0.7   & 56.47 & 56.15 &-0.32  \\
    6--10  & 121 & 47.9 & 52.1 & 5.2 & 1.4   & 26.11 & 28.01 & 1.90  \\
    11--15 & 118 & 37.2 & 62.8 & 3.2 & 2.7   & 22.85 & 23.70 & 0.85  \\
    16--20 & 238 & 32.4 & 67.6 & 3.2 & 13.9  & 20.40 & 20.81 & 0.41  \\
    21--25 & 266 & 25.2 & 74.8 & 3.1 & 70.1  & 20.89 & 22.51 & 1.62  \\
    26--30 & 152 & 22.4 & 77.6 & 2.4 & 162.6 & 20.92 & 22.30 & 1.38  \\
    \hline                                    
    1--30  & 986 & 32.1 & 67.9 & 3.3 & 48.1  & 21.66 & 22.63 & 0.97  \\
    \hline
  \end{tabular}
}\\
\subfloat[German-English\label{tab:results:german}] {
  \centering
  \footnotesize
  \begin{tabular}{|l|r|r|r|r|r|r|r|r|}
    \hline
    \multirow{2}{*}{Length} & \multirow{2}{*}{\#} & \multicolumn{4}{|c|}{Solve Stats} & \multicolumn{3}{|c|}{BLEU} \\
    & & \% eq & \% gt & \% err & ST & ReW & ILP & Diff \\
    \hline
    1--5   & 31  & 0.0 & 100.0 & 8.3 & 0.8   & 40.68 & 41.12 & 0.44  \\
    6--10  & 175 & 0.0 & 100.0 & 6.2 & 1.7   & 19.19 & 20.91 & 1.72  \\
    11--15 & 242 & 0.0 & 100.0 & 5.3 & 5.5   & 15.97 & 16.69 & 0.72  \\
    16--20 & 257 & 0.0 & 100.0 & 4.4 & 23.9  & 15.78 & 15.94 & 0.16  \\
    21--25 & 249 & 0.0 & 100.0 & 4.2 & 173.4 & 15.31 & 15.92 & 0.61  \\
    \hline                                   
    1--25  & 954 & 0.0 & 100.0 & 5.0 & 53.5  & 16.10 & 16.71 & 0.61   \\
    \hline
  \end{tabular}
}
\caption{\footnotesize Results on the two corpora.  Length: range of sentence lengths; \#: number of sentences in this range; \% eq: percentage of times ReWrite decoder and ILP decoder returned same model score; \% gt: percentage of times ILP decoder returned higher result than ReWrite; \% err: the micro-averaged percentage error between log model scores; ST: the average solve time per sentence of ILP decoder in seconds; BLEU ReW, BLEU ILP, BLEU Diff: the BLEU scores of the output and difference between BLEU scores.}  \label{tab:comparison}
\end{table*}

% TODO: difficult to decide how strong our claims can be regarding is
% ILP-M4 possible?
The ILP decoder produced output for 986 French sentences and 954
German sentences.  From this we can conclude that it is possible to
solve 90\% of our sentences exactly using ILP.  For the remaining 115
and 108 sentences we did not produce a solution due to: (1) the solver
not completing within 30 minutes, (2) the solver running out of
memory, or (3) the solver producing a one word output due to
(TODO:INSERT EXPLANATION).

Table~1 shows a comparison of the results obtained on the 986 French
and 954 German sentences using the ILP and ReWrite decoders.  The
results are broken down by sentence length range of the input
sentence.  We now turn our attention to the solve times obtained using
ILP (for the sentences for which the solution was found 30 minutes).
The table shows that the average solve time if under a minute per
sentence.  This longer sentences taking on average more time.  Thus
using the cutting planes method with ILP makes solving the ILP
programs tractable in practice (TODO: Really?).

% TODO: This next paragraph is strange, the percentage error in log
% space is strange in general!  And it is kind of out of the blue.
We next examine the model scores.  We can see in the French case the
ReWrite decoder finds the optimal solution 32.1\% of the time for
French and 0\% for German.  Although the same English string is
produced for 40.1\% of the French sentences and 29.1\% of the German
sentences\footnote{There may be precision and rounding errors in
  comparing the model scores, we call two log model scores equal if
  they are within 0.1 of one another}.  The percentage error rate
indicates how far away the ReWrite decoder's solution was from the
optimal in terms of log model score\footnote{We consider the
  percentage error of log scores because the numbers involved when
  converted to probabilities are very small resulting in large
  percentage errors.  Approximate macro-averaged percentage error of
  60\% for both corpora.}.  This number is a little misleading
although the percentage error rate is decreasing for longer sentences
in log space it is actually increasing in probability error space.  It
is difficult to interpret these percentage errors in either log space
or probability space.

Performing exact decoding increases the BLEU score by 0.97 points on
the French-English data set and 0.61 points on the German-English data
set.



% General averages:
% French: 
% 32.1\% of the time ReWrite and ILP equal in model score.
% 67.9\% of the time ILP beats ReWrite in model score.
% Log space stats:
% Micro averaged percentage error: 3.3\% (similar for macro)
% Maximum percentage error: 22.0\%
% Minimum percentage error (on the 67.1\%):  0.1\%
% Probability space stats:
% Micro averaged percentage error: 82.2\%
% Macro averaged percentage error: 60.6\%
% Maximum percentage error: 100.0\%
% Minimum percentage error: 21.5\%

% German: 
% 100\% of the time ILP beats ReWrite in model score.
% Log space stats:
% Micro averaged percentage error: 5.0\% (similar for macro)
% Maximum percentage error: 65.0\%
% Minimum percentage error:  0.7\%
% Probability space stats:
% Micro averaged percentage error: 95.3\%
% Macro averaged percentage error: 60.9\%
% Maximum percentage error: 100.0\%
% Minimum percentage error: 52.6\%



% \begin{table}[tp]
%   \centering
%   \begin{tabular}{|c|l|l|l|l|}
%     \hline
%     Length & Count & \%eq & \%R$<$ILP & Av.\%err \\
%     \hline
%     1--5 & 21 & 52.4 & 47.6 & 6.9 \\
%     6--10 & 121 & 47.9 & 52.1 & 5.2 \\
%     11--15 & 118 & 37.2 & 62.8 & 3.2 \\
%     16--20 & 238 & 32.4 & 67.6 & 3.2 \\
%     21--25 & 266 & 25.2 & 74.8 & 3.1 \\
%     26--30 & 152 & 22.4 & 77.6 & 2.4 \\
%     \hline 
%     1--30 & 986 & 32.1 & 67.9 & 3.3 \\
%     \hline
%   \end{tabular}
%   \caption{French comparisons}
%   \label{tab:french-compare}
% \end{table}

% \begin{table}[tp]
%   \centering
%   \begin{tabular}{|c|l|l|l|l|}
%     \hline
%     Length & Count & \%eq & \%R$<$ILP & Av.\%err \\
%     \hline
%     1--5 & 31 & 0.0 & 100.0 & 8.3 \\
%     6--10 & 175 & 0.0 & 100.0 & 6.2 \\
%     11--15 & 242 & 0.0 & 100.0 & 5.3 \\
%     16--20 & 257 & 0.0 & 100.0 & 4.4 \\
%     21--25 & 249 & 0.0 & 100.0 & 4.2 \\
%     \hline 
%     1--25 & 954 & 0.0 & 100.0 & 5.0 \\
%     \hline
%   \end{tabular}
%   \caption{German comparisons}
%   \label{tab:german-compare}
% \end{table}

% Length - length range of sentence (inclusive).
% Count - number of sentences in this length range.
% \%eq - percentage of time outputs had equal log model score (precision 0.1)
% \%R$<$ILP - percentage of time ILP had higher log model score than ReWrite.
% Av. \%err - average percentage error between the log model scores

% \begin{table}[tp]
%   \centering
%   \begin{tabular}{|c|l|l|l|l|}
%     \hline
%     Length & ReWrite BLEU & ILP BLEU & Diff & ILP ST \\
%     \hline
%     1--5 & 56.47 & 56.15 & -0.32 & 0.7 \\
%     6--10 & 26.11 & 28.01 & 1.90 & 1.4 \\
%     11--15 & 22.85 & 23.70 & 0.85 & 2.7 \\
%     16--20 & 20.40 & 20.81 & 0.41 & 13.9 \\
%     21--25 & 20.89 & 22.51 & 1.62 & 70.1 \\
%     26-30 & 20.92 & 22.30 & 1.38 & 162.6 \\
%     \hline 
%     1--30 & 21.66 & 22.63 & 0.97 & 48.1 \\
%     \hline
%   \end{tabular}
%   \caption{French Analysis}
%   \label{tab:french-analysis}
% \end{table}

% \begin{table}[tp]
%   \centering
%   \begin{tabular}{|c|l|l|l|l|}
%     \hline
%     Length & ReWrite BLEU & ILP BLEU & Diff & ILP ST \\
%     \hline
%     1--5 & 40.68 & 41.12 & 0.44 & 0.8 \\
%     6--10 & 19.19 & 20.91 & 1.72 & 1.7 \\
%     11--15 & 15.97 & 16.69 & 0.72 & 5.5 \\
%     16--20 & 15.78 & 15.94 & 0.16 & 23.9 \\
%     21--25 & 15.31 & 15.92 & 0.61 & 173.4 \\
%     \hline 
%     1--25 & 16.10 & 16.71 & 0.61 & 53.5 \\
%     \hline
%   \end{tabular}
%   \caption{German Analysis}
%   \label{tab:german-analysis}
% \end{table}

% Diff - BLEU difference between ILP and ReWrite
% ILP ST - average ILP solve time per sentence in seconds.



%%% Local Variables: 
%%% mode: latex
%%% TeX-master: "ilp-mt"
%%% End: 


\section{Discussion and Conclusions}
\label{sec:conclusion}
In this paper we have demonstrated that optimal decoding of IBM Model
4 is more practical than previously suggested.  Our results and
analysis show that exact decoding has a practical purpose.  It has
allowed us to investigate and validate the performance of the ReWrite
decoder through comparison of the outputs and model scores from the
two decoders.  Exact inference also provides an improvement in
translation quality as measured by BLEU score.

During the course of this research we have encountered numerous
challenges that were not apparent at the start.  These challenges
raise some interesting research questions and practical issues one
must consider when embarking on exact inference using ILP.  The first
issue is that the generation of the ILP programs can take a long time.
This leads us to wonder if there may be a way to provide tighter
integration of program generation and solving.  Such an integration
would avoid the need to query the models in advance for \emph{all}
possible model components the solver may require.

Related to this issue is how to tackle the incorporation of higher
order language models. Currently we use our bigram language model in a
brute-force manner: in order to generate the ILP we evaluate the
probability of all possible bigrams of English candidate tokens in
advance. It seems clear that with higher order models this process
will become prohibitively expensive. Moreover, even if the ILP could
be generated efficiently, they will obviously be larger and harder to
solve than our current ILPs. One possible solution may be the use of
so-called delayed column generation strategies which incrementally add
parts of the objective function (and hence the language model), but
only when required by the ILP solver.\footnote{Note that delayed
  column generation is dual to performing cutting planes.}

The use of ILP in other NLP tasks has provided a principled and
declarative manner to incorporate global linguistic constraints on the
system output.  This work lays the foundations for incorporating
similar global constraints for translation.  We are currently
investigating linguistic constraints for IBM Model~4 and other
word-based models in general.  A further extension is to reformulate
higher-level MT models (phrase- and syntax-based) within the ILP
framework.  These representations could be more desirable from a
linguistic constraint perspective as the formulation of constraints
may be more intuitive.


%%% Local Variables: 
%%% mode: latex
%%% TeX-master: "ilp-mt"
%%% End: 


\bibliographystyle{acl}
\small
\bibliography{ilp-mt}

\end{document}

