
A formula is local if its groundings relate any number of observed ground atoms 
to exactly one hidden ground atom.  For example, a grounding of the local 
formula \[lemma(p,+l_1) \wedge lemma(a,+l_2) \Rightarrow hasRole(p,a)\]
connects a hidden \emph{hasRole/2} ground atom to two observed \emph{plemma/2} 
ground atoms. This formula can be interpreted as the feature for the predicate 
and argument lemmas in the argument identification stage of a pipeline SRL 
system.
Note that the ``+'' prefix indicates that there is a different weight for each 
possible pair of lemmas $(l_1,l_2)$.

We divide our local formulae into four sets, one for each hidden predicate.  For 
instance, the set for \emph{argument/1} only contains formulae in which the hidden 
predicate is \emph{argument/1}. 

The sets for \emph{argument/1} and \emph{sense/2} predicates have similar 
formulae since each predicate only involves one token at time: the SRL argument 
or the SRL predicate token. The formulae in these sets are defined using only 
\emph{token} or \emph{extended} observed predicates. 

There are two differences 
between the  \emph{argument/1} and \emph{sense/2} formulae.  First, the \emph{argument/1} formulae use 
the \emph{possibleArg/1} predicate as precondition, while the sense formulae are conditioned on the
\emph{predicate/1} predicate. For instance, consider the \emph{argument/1} 
formula based on word forms: \[word(a,+w) \land possibleArg(a) \Rightarrow 
argument(a),\] and the equivalent version for the \emph{sense/2} predicate: \[word(p,+w) 
\land predicate(p) \Rightarrow sense(p,+s).\] This means we only apply the 
\emph{argument/1} formulae if the token is a potential SRL argument, and 
the \emph{sense/2} formulae if the token is a SRL predicate. 

The second difference is the fact that for the \emph{sense/2} formulae we have different weights for each possible sense (as indicated by the $+s$ term in the second formula above), while for the \emph{argument/1} formulae this is not the case. This follows naturally from the fact that  \emph{argument/1}  do not explicitly consider senses. 

% For example, the word formulae presented above  formulae, the index for the
%irst formula is $(+w)$, which is the orthography, while in the second formula 
%has as indexes $(+w,+s)$, which is the orthography and the sense label.

%What follows is a summary of the formulae in the \emph{sense/2} sets (we omit 
%the \emph{predicate/1} precondition on the formulae, this is equivalent to 
%conjoin the \emph{predicate(t)} in the formulae):
Table \ref{tbl:f1} presents templates for the local formuale of 
\emph{isArgument/1} and \emph{sense/2}. Templates allow us to compactly describe the 
FOL clauses of a ML. The template column shows the body of a clause. The last two columns of the table indicate if there is a clause with the given body and $isArgument(i)$ (I) or  $sense(i,+s)$ (S) head, respectively. For example, consider the first row: 
since the last two columns of the row are marked, this template expands into two 
formulae:
$\word(p,+w) \Rightarrow \isArgument(p)$ and $\word(p,+w) \Rightarrow \sense(p,+s)$.  
Including the preconditions for each hidden predicate we obtain the following 
formulae:
\[\mathit{possibleArg}(a) \land \mathit{word}(p,+w) \Rightarrow \mathit{isArgument}(p)\] and \[predicate(p) 
\land word(p,+w) \Rightarrow sense(p,+s).\]  In the case of the template marked 
with a ``*'' sign, the parameters $\mathbf{P}$ and $\mathbf{I}$, where $\mathbf{P} \in \{ppos,plemma\}$ and 
$\mathbf{I} \in \{-2,-1,0,1,2\}$, have to be replaced by any combination of possible values. Since we generate 
$isArgument$ and $sense$ formulae for this template, the row corresponds to $20$ formulae in total.
\begin{table}
\centering
\begin{tabular}{|p{6cm}|c|c|}\hline
   Template       & I & S \\\hline
   $ word(i,+w)$  & X & X \\
   $ \mathbf{P}(i+\mathbf{I},+v)$*  & X & X \\
   $ cpos(i+1,+c_1) \land cpos(i-1,+c_2)$ & X & X\\
   $ cpos(i+1,+c_1) \land cpos(i-1,+c_2) \land cpos(i+2,+c_3) \land 
   cpos(i-2,+c_4)$ & X & X\\
   $ dep(i,\_,+d)$ & X & X\\
   $ dep(\_,i,+d)$ & X & X\\
   $ ppos(i,+o)  \land dep(i,j,+d)$ & X & X \\
   $ ppos(i,+o_1)  \land ppos(j,+o_2) \land dep(i,j,+d)$ & X & X \\
   $ ppos(j,+o_1)  \land ppos(k,+o_2) \land dep(j,k,\_) \land dep(k,i,+d)$ & X & 
   X \\
   $ plemma(i,+l) \land dep(j,i,+d)$ & X & X\\
   $ frame(i,+f)$ & X & X\\
   (Empty Body)             &   & X \\
\hline
\end{tabular}
\caption{Templates of the local formulae for \emph{isArgument/1} and 
\emph{sense/2}. I: head of clause is $isArgument(i)$, S: head of clause is 
$sense(i,+s)$}
\label{tbl:f1}
\end{table}

Table \ref{tbl:f2} shows the local formuale for \emph{hasRole/2} and 
\emph{role/3} predicates,  for these formulae we use \emph{token}, 
\emph{extended} and \emph{path} predicates. In this case, these templates have 
as precondition the formula $predicate(p) \land possibleArg(a)$. This ensures that
the formulae are only applied for SRL predicates and potential SRL arguments.  
In the table we include the values to replace the template parameters with. Some 
of these formulae capture a notion of distance between SRL predicate and SRL 
argument and are implicitely conjoined with a $distance(p,a,+d)$ atom. If a formulae exists
both with and without $distance$ atom, we write \emph{Both} in the ``Dist'' column; if it only exists
with the $distance$ atom, we write \emph{Only}, otherwise \emph{No}.
\begin{table*}[th]
\centering
\begin{tabular}{|>{\small}p{10cm}|>{\small}c|>{\small}c|>{\small}c|>{\small}c|}\hline
 Template               & Parameters & Dist. & H & R \\\hline
   $ \mathbf{P}(p,+v)         $    & $\mathbf{P} \in S_1$   & Both & X & X \\
   $ plemma (p,+l) \land ppos(a,+o) $   & & No    & X &  \\
   $ ppos(p,+o)    \land plemma(a,+l)  $   & & No    & X &  \\
   $ plemma(p,+l_1) \land plemma(a,+l_2)$   & & Only  & X & X \\
   $ ppos(p,+o_1) \land ppos(a,+o_2)$       & & Only  & X &   \\
   $ ppos(p,+o_1) \land ppos(a+\mathbf{I},+o_2)$     & $\mathbf{I} \in 
   \{-1,0,1\}$ & Only& X &   \\
   $ lemma(p,+l)$ & & Only & & X  \\
   $ voice(p,+e) \land lemma(a,+l)$ & & Only & & X  \\
   $ cpos(p,+c_1) \land cpos(p+\mathbf{I},+c_2) \land cpos(a,+c_3) \land 
   cpos(a+\mathbf{J},c_4)$ & $\mathbf{I},\mathbf{J} \in \{-1,1\}^2$ & No & X & 
   X\\
   $ ppos(p,+v_1) \land ppos(a,\text{IN}) \land dep(a,m,\_) \land \mathbf{P}(m,+v_2) $ & $\mathbf{P} \in S_1$ & No & X &  X \\
   $ plemma(p,+v_1) \land ppos(a,\text{IN}) \land dep(a,m,\_) \land ppos(m,+v_2) $ & & No & X &  X \\
   $ \mathbf{P}(p,a,+v)         $    & $\mathbf{P} \in S_2$ & No & X & X \\
   $ \mathbf{P}(p,a,+v) \land plemma(p,+l) $ & $\mathbf{P} \in S_3$           & 
   No  & X & X \\
   $ \mathbf{P}(p,a,+v) \land plemma(p,+l_1) \land plemma(a,+l_2) $    &   
   $\mathbf{P} \in S_4$ & No  & X & X \\
   $ pathFrame(p,a,+t) \land plemma(p,+l) \land voice(p,+e) $ &      & No      & 
   X & X \\
   $ pathFrameDist(p,a,+t) $      &  & Only   & X & X \\
   $ pathFrameDist(p,a,+t) \land voice(p,+e) $ &  & Only           & X & X 
   \\
   $ pathFrameDist(p,a,+t) \land plemma(p,+l)$ &  & Only           & X & X 
   \\
   $ \mathbf{P}(p,a,+v) \land plemma(a,+l) $ & $\mathbf{P} \in S_5$ & Only         
   & X & X \\
   $ \mathbf{P}(p,a,+v) \land ppos(p,+o) $ & $\mathbf{P} \in S_5$   & Only        
   & X & X \\
   $ pathFrameDist(p,a,+t) \land ppos(p,+o_1) \land ppos(a,+o_2) $ &  & Only           
   & X & X \\
   $ path(p,a,+t) \land plemma(p,+l) \land cpos(a,+c) $ &  & Only          & X & 
   X \\
   $ dep(\_,a,+d)$  & & Only  & X & X \\
   $ dep(\_,a,+) \land voice(p,+e)$    &  & Only & X & X  \\
   $ dep(\_,a,+d_1) \land dep(\_,p,+d_2)$    & & Only & X & X  \\
   $ (Empty Body)                $    &                           & No & X & X \\
\hline
\end{tabular}


\caption{Templates of the local formulae for \emph{hasRole/2} and 
\emph{role/3}. H: head of clause is $hasRole(p,a)$, R: head of clause is 
$role(p,a,+r)$ and $S_1 = \{ppos,plemma\}, S_2=\{frame, unlabelFrame, path\}, S_3= \{frame,pathFrame\},S_4=\{frame,pathFrame,path\}, S_5= \{pathFrameDist, path\} $}
\label{tbl:f2}
\end{table*}

Note that Tables \ref{tbl:f1} and \ref{tbl:f2} do not mention the feature information provided in the corpora because 
this information was not available for every language. We therefore group the 
formulae which consider the \emph{feature/3} predicate into another a set we call 
\emph{feature} formulae. This is the summary of these formulae:
\begin{tabular}{p{7.5cm}}
   $ feat(p,+f,+v) \Rightarrow sense(p,+s)    $\\
   $ feat(p,+f,+v) \Rightarrow isArgument(a)    $\\
   $ feat(p,+f,+v1) \land feat(p,f,+v2) \Rightarrow $\\$\ \ hasRole(p,a)    $\\
   $ feat(p,+f,+v1) \land feat(p,f,+v2) \Rightarrow $\\$\ \ role(p,a,+r)   $\\
\end{tabular}


Additionally, we define a set of language specific formulae. They are aimed to 
capture the relations between argument and its siblings for the \emph{hasRole/2} 
and \emph{role/3} predicates.  In practice in turned out that these formulae were only beneficial for 
the Japanese language.  This is a summary of such formulae which we called 
\emph{argument siblings}:
\begin{tabular}{p{7.5cm}}
   $ dep(a,h,\_) \land dep(h,c,\_) \land ppos(a,+p1) \land $\\$\ \ ppos(c,+p2) 
   \Rightarrow hasRole(p,a)    $\\
   $ dep(a,h,\_) \land dep(h,c,\_) \land ppos(a,+p1) \land $\\$\ \ ppos(c,+p2) 
   \Rightarrow role(p,a,+r)    $\\
   $ dep(a,h,\_) \land dep(h,c,\_) \land plemma(a,+p1) \land $\\$\ \ ppos(c,+p2) 
   \Rightarrow hasRole(p,a)    $\\
   $ dep(a,h,\_) \land dep(h,c,\_) \land plemma(a,+p1) \land $\\$\ \ ppos(c,+p2) 
   \Rightarrow role(p,a,+r)    $\\
\end{tabular}


With these sets of formulae we can build specific MLNs for each language in the 
shared task. We group the formulae into the modules: \emph{isArgument/1}, 
\emph{hasRole/2}, \emph{role/3}, \emph{sense/3}, \emph{feature} and 
\emph{argument siblings}.  Table \ref{tbl:diff} shows the different 
configurations of such modules that we used for the individual languages.  We 
omit to mention the \emph{argument/1}, \emph{hasRole/2} and \emph{role/3} 
modules because they are present for all languages. 

% SR: This seems redundant given what was previously written 
% The \emph{feature} set corresponds to the FEAT 
% column provided in the corpus.  The presence of this set is determined by the 
% availability of this information in the corpus.  The presence of the 
% \emph{sense/2} set is determined by the labelling of senses in the corpora.  
% Finally, the formulae for the \emph{argument siblings} was implemented for the 
% Japanese during development. 

\begin{table}
\begin{center}
\small
\begin{tabular}{|l|c|c|c|}\hline
    Set         & Feature   & \emph{sense/2}  & Argument \\
                &            &        & siblings  \\\hline
Catalan         &   Yes      &  Yes   &  No  \\
Chinese         &   No       &  Yes   &  No  \\
Czech           &   Yes      &  No    &  No  \\
English         &   No       &  Yes   &  No  \\
German          &   Yes      &  Yes   &  No  \\
Japanese        &   Yes      &  No    &  Yes \\
Spanish         &   Yes      &  Yes   &  No  \\
\hline
\end{tabular}
\caption{Different configuration of the modules for the formulae of the 
languages.}
\label{tbl:diff}
\normalsize
\end{center}
\end{table}

A more detailed description of the formulae can be found in our MLN model files.\footnote{\url{http://thebeast.googlecode.com/svn/mlns/conll09}} They can be 
used both as a reference and as input to our Markov Logic Engine,\footnote{\url{http://thebeast.googlecode.com}} and thus allow the reader 
to easily reproduce our results.





