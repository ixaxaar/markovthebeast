%
% File naaclhlt2009.tex
%
% Contact: nasmith@cs.cmu.edu

\documentclass[11pt]{article}
\usepackage{naaclhlt2009}
\usepackage{times}
\usepackage{latexsym}
\usepackage[round]{natbib} 
\usepackage{amsmath}
\usepackage{amssymb} 
\usepackage[pdftex]{graphicx} 
\usepackage{hyperref} 


\setlength\titlebox{6.5cm}    % Expanding the titlebox

\title{Joint SRL with Markov Logic}

\author{}
%Sebastian Riedel \And Ivan Meza-Ruiz\\
%  Institute for Communicating and Collaborative Systems\\
%  School of Informatics\\
%  University of Edinburgh, Scotland\\
%  {\tt\{S.R.Riedel,I.V.Meza-Ruiz\}@sms.ed.ac.uk} }

\date{}

\begin{document}

%%% Local Variables: 
%%% mode: latex
%%% TeX-master: "paper"
%%% End: 

\newcommand{\R}{\mathcal{R}}
\newcommand{\T}{\mathcal{T}}
\newcommand{\bv}{\mathbf{v}}
\newcommand{\score}{s}
\newcommand{\obs}{\x_{o}}
\newcommand{\imp}{\Rightarrow}
\newcommand{\equi}{\Leftrightarrow}
\newcommand{\boldc}{\mathbf{c}}
\newcommand{\boldv}{\mathbf{v}}


\newcommand{\w}{\mathbf{w}}
\newcommand{\Y}{\mathcal{Y}}
\newcommand{\Yvar}{\mathbf{Y}}
\newcommand{\F}{F}
\newcommand{\opti}{\hat{\x}_{h}}
\newcommand{\f}{\mathbf{f}}
\newcommand{\x}{\mathbf{x}}
\newcommand{\y}{\mathbf{y}}
\newcommand{\ybest}{\hat{\y}}
\newcommand{\h}{\mathbf{h}}
\newcommand{\X}{\mathbf{X}}
\newcommand{\D}{\mathcal{D}}
\newcommand{\argmax}[1]{\underset{#1}{\text{arg max}}}
\newcommand{\guess}{\y'}
\newcommand{\Real}{\mathbb{R}}
\newcommand{\Int}{\mathbb{N}}
\newcommand{\bilevel}{\left\langle \Gamma,M\right\rangle }
\newcommand{\vocab}{\text{Vocabulary}}
\newcommand{\reduct}{\text{Reduct}}
\newcommand{\lforall}{\dot{\forall}}
\newcommand{\sscore}{\varsigma_{M}}
\newcommand{\cand}{\text{Cand}}
\newcommand{\unroll}{\text{Unroll}}
\newcommand{\ground}{\text{Ground}}
\newcommand{\mater}{\text{Materialize}}
\newcommand{\inner}{\text{Inner}}
\newcommand{\mapmodel}{\hat{N}}
\newcommand{\prob}{p}
\newcommand{\separate}{\text{Separate}}
\newcommand{\solve}{\text{solve}}
\newcommand{\MAP}{\text{MAP}}
\newcommand{\boldG}{\mathbf{G}}
\newcommand{\round}{\text{Round}}
\newcommand{\fracsolve}{\text{fractional-solve}}
\newcommand{\fractionals}{\text{Fractionals}}
\newcommand{\atoms}{\text{Atoms}}
\newcommand{\aux}{\lambda}
\newcommand{\smokes}{\forall x.\forall y.friends\left(x,y\right)\wedge smokes\left(x\right)\Rightarrow smokes\left(y\right)}
\newcommand{\data}{\mathcal{D}}
\newcommand{\I}{\mathbb{I}}
\newcommand{\Cliques}{\mathcal{C}}
\newcommand{\weightedsmokes}{\forall x.\forall y.\left(friends\left(x,y\right)\wedge smokes\left(x\right)\Rightarrow smokes\left(y\right)\left[w_{smokes}\right]\right)}
\newcommand{\ilpy}{\mathbf{a}}
\newcommand{\powerset}{\mathcal{P}}
\newcommand{\innersmokes}{friends\left(x,y\right)\wedge smokes\left(x\right)\Rightarrow smokes\left(y\right)}
\newcommand{\annasmokes}{smokes\left(Anna\right),friends\left(Anna,Peter\right)}
\newcommand{\finitevocab}{\left\{  \left\{  smokes,friends\right\}  ,\left\{  \right\}  ,\left\{  Anna,Peter\right\}  ,\left\{  x,y\right\}  \right\}  }
\newcommand{\determin}{\text{Deterministic}}
\newcommand{\nondeter}{\text{Nondeterministic}}




\maketitle
\begin{abstract}
In this paper we present a Markov Logic Network for Semantic Role
Labelling that jointly performs frame disambiguation, argument
identification and argument classification. When compared to a
pipeline model with the same feature set, we observe significant
improvements, but only for out-of-domain data. This suggests that
joint modelling can be particular helpful in out-of-domain
scenarios. We validate this hypothesis by artificially removing
features from the in-domain.  

Blah!
\end{abstract}

\section{Introduction}

The search for the most likely translation (aka decoding, search,
MAP inference) given a statistical MT model has traditionally been
peformed by greedy or beam-based methods/heuristics {[}site some{]}.
While being efficient, most of these methods (do we know enough about
decoding to make any claims here?) have two drawbacks: (1) they are
approximate and give no bounds as to how far their solution is away
from the true optimum---this makes it sometimes hard to tell whether
the model or the search algorithm need to be improved; (2) they don't
scale (development-wise, not efficience) well with additional global
constraints. 

By constrast, recently researchers in other NLP appliciation areas
have increasingly relied on Integer Linear Programming as means to
find MAP solutions. ILP overcomes the two drawbacks mentioned above
as is guaranteed to be exact, and can be easily used to enforce global
constraints through additional linear inequalities. However, guaranteed
exactness, and generic global inference comes usually at the price
of efficiency. 

In fact, ILP has also been used as a method for MT decoding with IBM
Model 4 {[}cite Germann{]}. In {[}cite{]} work it became clear that
a naive ILP-based does not scale up to more than simple short sentences
due to an exponential number of constraints necessary to represent
the decoding problem in ILP. However, recent work in Dependency Parsing
{[}cite you and me{]} showed that ILP can still be efficient for very
large prograns when used in an incremental fashion. This raises the
question whether incremental/Cutting Plane ILP can also be used for
decoding of Model 4 for real world sentences. 

In this work we show that it is indeed possible to decode Model 4
with incremental ILP, at least for sentences to up to 30(? we have
to be more defensive here I think) words, and a simple two-gramm languange
model. This allowed us to give a definite answer as to how good/bad
Model 4 actually is, and how good/bad its state-of-the-art rewrite
decoder {[}cite{]}. Exact inference increases Bleu score by about
1 point for two language pairs when compared to the results of the
rewrite decoder {[}this is not really the answer{]}. From this we
conclude that the rewrite decoder indeed performs well, but still
can be slightly improved. (At the price of efficiency, that is)

Moreover, we believe that our work can be the basis of exciting and
really cool and also sexy future work. For example, as mentioned ILP
allows for a principled and declarative implementation of global constraints
and hence we may ask whether Model 4 can be improved through additional
global linguistic constraints; can we map other MT problems to an
ILP representation, and will this get slower, or maybe faster {[}why
could it{]}? How can larger language models be handled? (such models
are relatively easy to incorporate in left to right or greedy approaches
but pose a serious problem in the ILP sense). 

Finally, we observed that preprocessing and generating of ILPs is
as cpu-intensive as actual inference. This leads to the more general
question of how to integrate model construction more tightly with
the ILP solver (maybe by using Pricing strategies).

\section{Markov Logic} \label{sec:markovlogic}

Markov Logic (ML) \cite{richardson05markov} is a Statistical Relational Learning language based on First Order Logic and Markov Networks. It can be seen as a formalism that extends First Order Logic to allow formulae that can be violated
with some penalty. From an alternative point of view, it is an expressive
template language that uses First Order Logic formulae to instantiate
Markov Networks of repetitive structure. 

Let us describe Markov Logic by considering the predicate identification task. In Markov Logic we can model this task by first introducing a set of logical predicates\footnote{In the cases were is not obvious whether we refer to SRL or ML predicates we add the prefix SRL or ML, respectively.} such as \emph{isPredicate(Token)} or \emph{word(Token,Word)}. Then we specify a set of weighted first order formulae that define a distribution over sets of ground atoms of these predicates (or so-called \emph{possible worlds}). 

Ideally, the distribution we define with these weighted formulae assigns high probability to possible worlds where SRL predicates are correctly identified and a low probability to worlds where this is not the case. For example, a suitable set of weighted formulae would assign a high probability to the world\footnote{``Haag plays Elianti'' is a segment of a sentence in training corpus.}
\begin{eqnarray*}
 &\{ word\left(1,Haag\right),word(2,plays),\\
 & word(3,Elianti),isPredicate(2) \}& \end{eqnarray*}
and a low one to
\begin{eqnarray*}
& \{ word\left(1,Haag\right),word(2,plays),\\
 & word(3,Elianti),isPredicate(3) \} &\end{eqnarray*}
In Markov Logic a set $M=\left\{ \left(\phi,w_{\phi}\right)\right\} _{\phi}$ of weighted first order formulae is called a \emph{Markov Logic Network}~(MLN). It assigns the probability
\begin{equation}
\prob\left(\y\right)=\frac{1}{Z}\exp\left(
\sum_{\left(\phi,w\right)\in M} w
\sum_{\boldc\in C^{n_{\phi}}}f_{\boldc}^{\phi}\left(\y\right)
\right)
\label{eq:prob}
\end{equation}
to the possible world $\y$. Here $f_{\boldc}^{\phi}$ is a feature
function that returns 1 if in the possible world $\y$ the ground
formula we get by replacing the free variables in $\phi$ by the constants
in $\boldc$ is true and 0 otherwise. $C^{n_{\phi}}$ is the set
of all tuples of constants we can replace the free variables in $\phi$
with. $Z$ is a normalisation constant. Note that this distribution corresponds to a Markov Network where nodes represent ground atoms and factors represent ground formulae.

For example, if $M$ contains the formula $\phi$ \[
word\left(x,'take'\right)\Rightarrow isPredicate\left(x\right)\]
then its corresponding log-linear model has, among others, a feature 
$f_{t1}^{\phi}$ for which $x$ in $\phi$ has been replaced by the constant $t_1$ and that returns 1 if \[
word\left(1,'take'\right)\Rightarrow isPredicate\left(1\right)\]
is true in $\y$ and 0 otherwise.

We will refer predicates such as \emph{word} as \emph{observed} because they are known in advance. In contrast, \emph{isPredicate} is \emph{hidden} because we need to infer it at test time.




\section{Systems} \label{sec:systems} 
%\input{system.tex}

\subsection{Pipeline}\label{sec:pipeline} 
 
%\input{pipeline.tex}

\subsection{Joint}\label{sec:joint} 

%\input{joint.tex}

\section{Inference and Learning}\label{sec:inference}

%%\subsection{Learning}

%An MLN we use to model the collective SRL task is presented in section \ref{sec:model}. We learn the weights associated this MLN using 1-best MIRA~\citep{crammer01ultraconservative} Online Learning method. 
%%Note that we only consider formulae that appear more than once in the training set.

%\subsection{Inference}

%%Assuming that we have an MLN with a set of suitable weights and a given sentence then we need to predict the choice of predicates, frame types, arguments and role labels with maximal a posteriori probability. To this end we apply a method that is both exact and efficient: Cutting Plane Inference~\citep[CPI,][]{riedel08improving} with Integer Linear Programming~(ILP) as base solver. We use it for inference at test time as well as during the MIRA online learning process.

Assuming that we have an MLN, a set of weights and a given sentence then we need to predict the choice of predicates, frame types, arguments and role labels with maximal a posteriori probability. To this end we apply a method that is both exact and efficient: Cutting Plane Inference~\cite[CPI,][]{riedel08improving} with Integer Linear Programming~(ILP) as base solver. We use it for inference at test time as well as during the MIRA online learning process.




\section{Results}\label{sec:results}

%In this section we describe our experimental setup and results.

\subsection{Experimental setup}
\label{sec:experimental-setup}

Our experimental setup is designed to answer several questions: (1) Is
exact inference in IBM Model 4 possible for sentences of moderate
length? (2) How fast is exact inference using the cutting plane
method? (3) How well does the ReWrite decoder perform in terms of
finding the optimal solution? (4) Does exact decoding increase provide
better translations?

In order to answer these questions we obtain a trained IBM Model 4 for
French-English and German-English on Europarl version 3 using GIZA++.
A bigram language model with Witten-Bell smoothing was built using the
CMU-Cambridge Language Modeling Toolkit.

For exact decoding we use the two models to generate ILP programs for
sentences of up to (and including) length 30 tokens for French and 25
tokens for German\footnote{These limits were imposed to ensure the
  Python script generating the ILP programs did not run out of
  memory.}.  We filter translation candidates in a similar manner to
(TODO:CITE) by using only the top ten translations for each
word\footnote{Based on $t(e|f)$.} and a list of zero fertility
words\footnote{Extracted using the rules in the filter script
  \texttt{rewrite.mkZeroFert.perl}}.  This resulted in 1101 French
sentences and 1062 German sentences for testing purposes.  The ILP
programs were then solved using the method described in
Section~\ref{sec:ilp}.  The same models were used with the ISI ReWrite
Decoder to solve the same set of sentences.


\subsection{Results}
\label{sec:results-results}

\begin{table*}[t]
  \centering
  \subfloat[French-English\label{tab:results:french}] {
    \footnotesize
  \centering
  \begin{tabular}{|l|r|r|r|r|r|r|r|r|}
    \hline
    \multirow{2}{*}{Length} & \multirow{2}{*}{\#} & \multicolumn{4}{|c|}{Solve Stats} & \multicolumn{3}{|c|}{BLEU} \\
    & & \% eq & \% gt & \%err & ST & ReW & ILP & Diff \\
    \hline
    1--5   & 21  & 52.4 & 47.6 & 6.9 & 0.7   & 56.47 & 56.15 &-0.32  \\
    6--10  & 121 & 47.9 & 52.1 & 5.2 & 1.4   & 26.11 & 28.01 & 1.90  \\
    11--15 & 118 & 37.2 & 62.8 & 3.2 & 2.7   & 22.85 & 23.70 & 0.85  \\
    16--20 & 238 & 32.4 & 67.6 & 3.2 & 13.9  & 20.40 & 20.81 & 0.41  \\
    21--25 & 266 & 25.2 & 74.8 & 3.1 & 70.1  & 20.89 & 22.51 & 1.62  \\
    26--30 & 152 & 22.4 & 77.6 & 2.4 & 162.6 & 20.92 & 22.30 & 1.38  \\
    \hline                                    
    1--30  & 986 & 32.1 & 67.9 & 3.3 & 48.1  & 21.66 & 22.63 & 0.97  \\
    \hline
  \end{tabular}
}\\
\subfloat[German-English\label{tab:results:german}] {
  \centering
  \footnotesize
  \begin{tabular}{|l|r|r|r|r|r|r|r|r|}
    \hline
    \multirow{2}{*}{Length} & \multirow{2}{*}{\#} & \multicolumn{4}{|c|}{Solve Stats} & \multicolumn{3}{|c|}{BLEU} \\
    & & \% eq & \% gt & \% err & ST & ReW & ILP & Diff \\
    \hline
    1--5   & 31  & 0.0 & 100.0 & 8.3 & 0.8   & 40.68 & 41.12 & 0.44  \\
    6--10  & 175 & 0.0 & 100.0 & 6.2 & 1.7   & 19.19 & 20.91 & 1.72  \\
    11--15 & 242 & 0.0 & 100.0 & 5.3 & 5.5   & 15.97 & 16.69 & 0.72  \\
    16--20 & 257 & 0.0 & 100.0 & 4.4 & 23.9  & 15.78 & 15.94 & 0.16  \\
    21--25 & 249 & 0.0 & 100.0 & 4.2 & 173.4 & 15.31 & 15.92 & 0.61  \\
    \hline                                   
    1--25  & 954 & 0.0 & 100.0 & 5.0 & 53.5  & 16.10 & 16.71 & 0.61   \\
    \hline
  \end{tabular}
}
\caption{\footnotesize Results on the two corpora.  Length: range of sentence lengths; \#: number of sentences in this range; \% eq: percentage of times ReWrite decoder and ILP decoder returned same model score; \% gt: percentage of times ILP decoder returned higher result than ReWrite; \% err: the micro-averaged percentage error between log model scores; ST: the average solve time per sentence of ILP decoder in seconds; BLEU ReW, BLEU ILP, BLEU Diff: the BLEU scores of the output and difference between BLEU scores.}  \label{tab:comparison}
\end{table*}

% TODO: difficult to decide how strong our claims can be regarding is
% ILP-M4 possible?
The ILP decoder produced output for 986 French sentences and 954
German sentences.  From this we can conclude that it is possible to
solve 90\% of our sentences exactly using ILP.  For the remaining 115
and 108 sentences we did not produce a solution due to: (1) the solver
not completing within 30 minutes, (2) the solver running out of
memory, or (3) the solver producing a one word output due to
(TODO:INSERT EXPLANATION).

Table~1 shows a comparison of the results obtained on the 986 French
and 954 German sentences using the ILP and ReWrite decoders.  The
results are broken down by sentence length range of the input
sentence.  We now turn our attention to the solve times obtained using
ILP (for the sentences for which the solution was found 30 minutes).
The table shows that the average solve time if under a minute per
sentence.  This longer sentences taking on average more time.  Thus
using the cutting planes method with ILP makes solving the ILP
programs tractable in practice (TODO: Really?).

% TODO: This next paragraph is strange, the percentage error in log
% space is strange in general!  And it is kind of out of the blue.
We next examine the model scores.  We can see in the French case the
ReWrite decoder finds the optimal solution 32.1\% of the time for
French and 0\% for German.  Although the same English string is
produced for 40.1\% of the French sentences and 29.1\% of the German
sentences\footnote{There may be precision and rounding errors in
  comparing the model scores, we call two log model scores equal if
  they are within 0.1 of one another}.  The percentage error rate
indicates how far away the ReWrite decoder's solution was from the
optimal in terms of log model score\footnote{We consider the
  percentage error of log scores because the numbers involved when
  converted to probabilities are very small resulting in large
  percentage errors.  Approximate macro-averaged percentage error of
  60\% for both corpora.}.  This number is a little misleading
although the percentage error rate is decreasing for longer sentences
in log space it is actually increasing in probability error space.  It
is difficult to interpret these percentage errors in either log space
or probability space.

Performing exact decoding increases the BLEU score by 0.97 points on
the French-English data set and 0.61 points on the German-English data
set.



% General averages:
% French: 
% 32.1\% of the time ReWrite and ILP equal in model score.
% 67.9\% of the time ILP beats ReWrite in model score.
% Log space stats:
% Micro averaged percentage error: 3.3\% (similar for macro)
% Maximum percentage error: 22.0\%
% Minimum percentage error (on the 67.1\%):  0.1\%
% Probability space stats:
% Micro averaged percentage error: 82.2\%
% Macro averaged percentage error: 60.6\%
% Maximum percentage error: 100.0\%
% Minimum percentage error: 21.5\%

% German: 
% 100\% of the time ILP beats ReWrite in model score.
% Log space stats:
% Micro averaged percentage error: 5.0\% (similar for macro)
% Maximum percentage error: 65.0\%
% Minimum percentage error:  0.7\%
% Probability space stats:
% Micro averaged percentage error: 95.3\%
% Macro averaged percentage error: 60.9\%
% Maximum percentage error: 100.0\%
% Minimum percentage error: 52.6\%



% \begin{table}[tp]
%   \centering
%   \begin{tabular}{|c|l|l|l|l|}
%     \hline
%     Length & Count & \%eq & \%R$<$ILP & Av.\%err \\
%     \hline
%     1--5 & 21 & 52.4 & 47.6 & 6.9 \\
%     6--10 & 121 & 47.9 & 52.1 & 5.2 \\
%     11--15 & 118 & 37.2 & 62.8 & 3.2 \\
%     16--20 & 238 & 32.4 & 67.6 & 3.2 \\
%     21--25 & 266 & 25.2 & 74.8 & 3.1 \\
%     26--30 & 152 & 22.4 & 77.6 & 2.4 \\
%     \hline 
%     1--30 & 986 & 32.1 & 67.9 & 3.3 \\
%     \hline
%   \end{tabular}
%   \caption{French comparisons}
%   \label{tab:french-compare}
% \end{table}

% \begin{table}[tp]
%   \centering
%   \begin{tabular}{|c|l|l|l|l|}
%     \hline
%     Length & Count & \%eq & \%R$<$ILP & Av.\%err \\
%     \hline
%     1--5 & 31 & 0.0 & 100.0 & 8.3 \\
%     6--10 & 175 & 0.0 & 100.0 & 6.2 \\
%     11--15 & 242 & 0.0 & 100.0 & 5.3 \\
%     16--20 & 257 & 0.0 & 100.0 & 4.4 \\
%     21--25 & 249 & 0.0 & 100.0 & 4.2 \\
%     \hline 
%     1--25 & 954 & 0.0 & 100.0 & 5.0 \\
%     \hline
%   \end{tabular}
%   \caption{German comparisons}
%   \label{tab:german-compare}
% \end{table}

% Length - length range of sentence (inclusive).
% Count - number of sentences in this length range.
% \%eq - percentage of time outputs had equal log model score (precision 0.1)
% \%R$<$ILP - percentage of time ILP had higher log model score than ReWrite.
% Av. \%err - average percentage error between the log model scores

% \begin{table}[tp]
%   \centering
%   \begin{tabular}{|c|l|l|l|l|}
%     \hline
%     Length & ReWrite BLEU & ILP BLEU & Diff & ILP ST \\
%     \hline
%     1--5 & 56.47 & 56.15 & -0.32 & 0.7 \\
%     6--10 & 26.11 & 28.01 & 1.90 & 1.4 \\
%     11--15 & 22.85 & 23.70 & 0.85 & 2.7 \\
%     16--20 & 20.40 & 20.81 & 0.41 & 13.9 \\
%     21--25 & 20.89 & 22.51 & 1.62 & 70.1 \\
%     26-30 & 20.92 & 22.30 & 1.38 & 162.6 \\
%     \hline 
%     1--30 & 21.66 & 22.63 & 0.97 & 48.1 \\
%     \hline
%   \end{tabular}
%   \caption{French Analysis}
%   \label{tab:french-analysis}
% \end{table}

% \begin{table}[tp]
%   \centering
%   \begin{tabular}{|c|l|l|l|l|}
%     \hline
%     Length & ReWrite BLEU & ILP BLEU & Diff & ILP ST \\
%     \hline
%     1--5 & 40.68 & 41.12 & 0.44 & 0.8 \\
%     6--10 & 19.19 & 20.91 & 1.72 & 1.7 \\
%     11--15 & 15.97 & 16.69 & 0.72 & 5.5 \\
%     16--20 & 15.78 & 15.94 & 0.16 & 23.9 \\
%     21--25 & 15.31 & 15.92 & 0.61 & 173.4 \\
%     \hline 
%     1--25 & 16.10 & 16.71 & 0.61 & 53.5 \\
%     \hline
%   \end{tabular}
%   \caption{German Analysis}
%   \label{tab:german-analysis}
% \end{table}

% Diff - BLEU difference between ILP and ReWrite
% ILP ST - average ILP solve time per sentence in seconds.



%%% Local Variables: 
%%% mode: latex
%%% TeX-master: "ilp-mt"
%%% End: 


\section{Conclusion} \label{sec:conclusion}

%

We presented a Markov Logic Network that performs joint multi-lingual Semantic 
Role Labelling. This network achieves the third best semantic F-scores in the 
closed track among the SRLOnly systems of the CoNLL-09 Shared Task, and sixth 
best semantic scores among SRLOnly and Joint systems for the closed task.
% SR closed or open???

We observed that the inclusion of features which take into account 
information about the siblings of the argument were benefitial for 
SRL performance on the Japanese dataset. We also noticed that our poor performance with Czech 
are caused by our frame ID heuristic. Further work 
has to be done in order to overcome this problem. 


\bibliographystyle{plainnat}
\bibliography{seb}


\end{document}
