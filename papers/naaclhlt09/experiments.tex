
For our experiments we modified the PropBank corpus of the CoNLL-05 shared task 
\citep{carreras05introduction} to be compatible with the format of CoNL-08 
shared task \citep{surdeanu08conll} which uses dependencies trees rather than 
constituents. We use the Charniak parses avaible in the original corpus and 
converted into dependencies using the softeare developed by 
\citet{johansson07conversion}. For indomain evaluation we use the WSJ test set, 
for the outdomain evaluation we use the Brown test set. 

This section describes the four systems on which we tested our ML model.
Firstly, we use the full model described in the previous section. Second, we 
eliminate some structural constrains to create a bottom-up which resembles a 
pipeline system, but which still a jointly model of the task. Third we eliminate 
the argument identification model from our previous models. Finally, we create a 
pipeline system to which we can compare against our joint models.

The \emph{full model} uses all the predicates, local formulae and global 
formulae. 

The \emph{bottom-up model} ommits the top-down structural constrain introduced 
in the section \ref{sec:global}. This results in a model which behaves like a 
pipeline system which aparently pases the decisions mad  


6.3 Full/Bottom-up model w/o isArg (what about pipeline w/o isArg)
6.4 Pipeline
